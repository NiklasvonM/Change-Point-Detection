\begin{lemma}
    Let $j \in \NN$ be some natural number. Then the inequality
    \begin{equation} \label{number of lattice elements inequality}
        \left| \{ x \in \ZZ^d \mid \|x\|_{\infty} = j \} \right| \leq d 2^d j^{d-1}
    \end{equation}
    holds.
\end{lemma}
\begin{proof}
    Two simple inductions show
    \begin{align*} 
        \left| \{ x \in \ZZ^d \mid \|x\|_{\infty} = j \} \right|
        & = \left| \{ x \in \ZZ^d \mid \|x\|_{\infty} \leq j \} \right| - \left| \{ x \in \ZZ^d \mid \|x\|_{\infty} \leq j-1 \} \right| \nonumber \\
        & = (2j+1)^d - (2j-1)^d  \nonumber \\
        & \leq d 2^d j^{d-1}.
    \end{align*}
\end{proof}

\begin{lemma} \label{for lemma 3: limit of psd mat}
    The limit of a sequence of symmetric, positive-semidefinite matrices is symmetric and positive-semidefinite.
\end{lemma}
\begin{proof}
    By \cite{boyd2004convex} Example 2.15, the set of symmetric, positive-semidefinite $(d \times d)$-matrices $\mathbf{S}_+^d$ is a proper cone. In particular, it is a closed set in the complete normed space $\RR^{d \times d}$. Hence any limit of elements in $\mathbf{S}_+^d$ lies in $\mathbf{S}_+^d$.

    Alternatively, one may show the statement directly: Let $(\Sigma(n))_n$ be a sequence of symmetric, positive-semidefinite $(d \times d)$-matrices with $\Sigma(n) = (\sigma{i, j}(n))_{i, j = 1, ..., d}$ that converges to some matrix $\Sigma = (\sigma_{i, j})_{i, j = 1, ..., d}$. The convergence $\Sigma(n) \to \Sigma$ is equivalent of the component-wise convergence. Now let $x \in \RR^d$ be any vector. Due to positive-semidefinitness of the matrices $\Sigma(n)$, we have
    \[ 0 \leq x^t \Sigma(n) x = \sum\limits_{i, j = 1}^d \sigma_{i, j}(n) x_i x_j \]
    for all $n$. As $(x^t \Sigma(n) x)_n$ is a sequence of non-negative numbers, its limit
    \[ \sum\limits_{i, j = 1}^d \sigma_{i, j} x_i x_j = x^t \Sigma x \]
    is also non-negative. Since $x$ was chosen arbitrarily, this shows the po\-si\-tive-semi\-de\-fi\-nit\-ness of $\Sigma$. The symmetry of $\Sigma$ is obvious.
\end{proof}


The following result is \cite{[0]BUCCHIA2017344} Lemma 3.
\begin{lemma} \label{lemma:3}
    Assume Assumptions \ref{assumption:weak_stationarity}, \ref{assumption:rho_mixing}, \ref{assumption:2+delta_moment_sup} and \ref{assumption:alpha1_summability} and let the random field $X$ be centered and have values in $\RR^k$ and let $S_n$ be the partial sum field of $X$. Define \[ \Gamma(n, \mathbf{t}) \coloneqq \Cov(S_n(\mathbf{t})). \]
    Then 
    \[ \lim\limits_{n \to \infty} \Gamma(n, \mathbf{t}) = [\mathbf{t}] \Gamma \]
    for all $\mathbf{t} \in [0, 1]^d$ with the long-run variance matrix 
    \[ \Gamma = \sum\limits_{\mathbf{v} \in \ZZ^d} \Cov(X_{\usbf[0]}, X_\mathbf{v}) \]
    of $X$ and the above series converges absolutely. Furthermore, $S_n$ converges weakly in $D_{\RR^k}([0, 1]^d)$ to a Brownian sheet in $\RR^k$ with covariance matrix $\Gamma$.
\end{lemma}
\begin{proof}
    We show the assumptions of Lemma \ref{lemma:2}. 
    
    \noindent Lemma \ref{lemma:2} a) is true because $X$ is centered and thus $S_n$ is, too.

    We now show the summability of the covariograms.
    It is remarked in \cite{[23]guyon1995random} p. 110 (in the context of univariate random fields) that the moment existence Assumption \ref{assumption:2+delta_moment_sup} implies
    \begin{equation*}
        |\Cov(X^i_{\usbf[0]}, X^i_{\mathbf{v}})| = |\gamma_{i, i}(\mathbf{v})| \leq 8 \sup\limits_{\mathbf{l}} \|X_\mathbf{l}^i\|_{2+\delta}^2 \alpha_{1, 1}^{X^i}(\|\mathbf{v}\|_\infty)^{\frac{\delta}{2+\delta}}
    \end{equation*}
    for all $i = 1, ..., k$ and large enough $\mathbf{v}$. 
    Since 
    \[ \alpha_{1, 1}^{X^i} \leq \alpha_{1, 1}^{X}, \] 
    according to \cite{[23]guyon1995random}, Assumption \ref{assumption:alpha1_summability} together with the weak stationarity Assumption \ref{assumption:weak_stationarity} then shows the summability of the covariogram $\gamma_{i, i}$.
    If we want to apply this argument to $\gamma_{i, j}$, we need to consider the random field $X^{j, i}$ by which we mean the $\RR$-valued random field $X^j$ for which $X_{\usbf[0]}^j$ is replaced by $X_{\usbf[0]}^i$. However, this random field also satisfies the Assumption \ref{assumption:2+delta_moment_sup} and
    \[ \alpha_{1, 1}^{X^{j, i}} \leq \alpha_{1, 1}^{X} \]
    as the $\sigma$-fields generated by this univariate random field are included in those generated by $X$.
    By Inequality \eqref{number of lattice elements inequality}, we have
    \begin{align*}
        \sum\limits_{\mathbf{v} \in \ZZ^d} |\gamma_{i, j}(\mathbf{v}) |
        & \leq \sum\limits_{\mathbf{v} \in \ZZ^d} 8 \sup\limits_{\mathbf{l}} \| X_{\mathbf{l}}^{j, i} \|_{2+\delta}^2 \alpha_{1, 1}^{X^{j, i}}(\|\mathbf{v}\|_\infty)^{\frac{\delta}{2+\delta}} + C \\
        & \leq 8 \sup\limits_{\mathbf{l}} \| X_{\mathbf{l}} \|_{2+\delta}^2 \sum\limits_{m \geq 1} \left( d 2^d m^{d-1} \right) \alpha_{1, 1}^{X}(m)^{\frac{\delta}{2+\delta}} + C
    \end{align*}
    for a constant $C$ that compensates for the fact that the covariance inequality only holds for large enough $\mathbf{v}$. Using Assumptions \ref{assumption:2+delta_moment_sup} and \ref{assumption:alpha1_summability}, we get
    \begin{equation} \label{lemma3:summierbarkeit gamma}
        \sum\limits_{\mathbf{v} \in \ZZ^d} |\gamma_{i, j}(\mathbf{v})| < \infty.
    \end{equation}

    Using the definition of $\Gamma$, the bilinearity of the cross-covariance and the weak stationarity of $X$, we obtain
    \begin{equation*}\begin{split}
        \Gamma(n, \mathbf{t})_{i, j}
        & = \Cov\left(n^{-d/2} \sum\limits_{\usbf[1] \leq \mathbf{m} \leq \floor{n \mathbf{t}}} X_\mathbf{m} \right)_{i, j} \\
        & = \frac{1}{n^{d}} \sum\limits_{\usbf[1] \leq \mathbf{m} \leq \floor{n \mathbf{t}}}  \sum\limits_{\usbf[1] \leq \mathbf{m}' \leq \floor{n \mathbf{t}}}  \Cov(X_{\mathbf{m}}^{i}, X_{\mathbf{m}'}^j) \\
        & = \frac{1}{n^{d}} \sum\limits_{-\floor{n \mathbf{t}} < \mathbf{v} < \floor{n \mathbf{t}}} |\{ (\mathbf{m}, \mathbf{m}') \in \{ \usbf[1], ..., \floor{n \mathbf{t}} \}^2 \mid \mathbf{m} - \mathbf{m}' = \mathbf{v} \}| \Cov(X_\mathbf{v}^i, X_{\usbf[0]}^j).
    \end{split}\end{equation*}
    A simple induction over $d$, using a second induction over $\floor{n t}$ for the base case $d=1$, shows
    \[ |\{ (\mathbf{m}, \mathbf{m}') \in \{ \usbf[1], ..., \floor{n \mathbf{t}} \}^2 \subset \ZZ^d \times \ZZ^d \mid \mathbf{m} - \mathbf{m}' = \mathbf{v} \}| = \prod\limits_{l=1}^d (\floor{n \mathbf{t}_l} - |\mathbf{v}_l|) \]
    for $-\floor{n \mathbf{t}} \leq \mathbf{v} \leq \floor{n \mathbf{t}}$. Hence
    \begin{align*}
        \Gamma(n, \mathbf{t})_{i, j}
        & = \frac{1}{n^{d}} \sum\limits_{-\floor{n \mathbf{t}} < \mathbf{v} < \floor{n \mathbf{t}}} \Cov(X_\mathbf{v}^i, X_{\usbf[0]}^j) \prod\limits_{l=1}^d (\floor{n \mathbf{t}_l} - |\mathbf{v}_l|) \\
        & = \sum\limits_{\mathbf{v} \in \ZZ^d} \mathbbm{1}_{|\mathbf{v}| < \floor{n \mathbf{t}}} \gamma_{i, j}(\mathbf{v}) \prod\limits_{l=1}^d \frac{\floor{n \mathbf{t}_l} - |\mathbf{v}_l|}{n}.
    \end{align*}
    As the sum
    \begin{align*}
        \sum\limits_{\mathbf{v} \in \ZZ^d} \gamma_{i, j}(\mathbf{v}) = \Gamma_{i, j}
    \end{align*}
    exists by \eqref{lemma3:summierbarkeit gamma}, we can, pointwise in $\omega$, apply the dominated convergence theorem w.r.t. the counting measure and the dominating function $[\mathbf{t}]\gamma_{i, j}$ to get
    \begin{align*}
            \lim\limits_{n \to \infty} \Gamma(n, \mathbf{t})_{i, j}
            & = \lim\limits_{n \to \infty} \sum\limits_{\mathbf{v} \in \ZZ^d} \mathbbm{1}_{|\mathbf{v}| < \floor{n \mathbf{t}}} \gamma_{i, j}(\mathbf{v}) \prod\limits_{l=1}^d \frac{\floor{n \mathbf{t}_l} - |\mathbf{v}_l|}{n} \\
            & = \sum\limits_{\mathbf{v} \in \ZZ^d} \gamma_{i, j}(\mathbf{v}) [\mathbf{t}] \\
            & = [\mathbf{t}] \Gamma_{i, j}.
    \end{align*}
    Thus the Assumption b) of Lemma \ref{lemma:2} is fulfilled. Applying Lemma \ref{for lemma 3: limit of psd mat}, we also see that $\Gamma$ is a symmetric, positive-semidefinite matrix. 

    By assumption the conditions of Lemma \ref{lemma:1} are fulfilled. \eqref{lemma1:(7)} therefore shows
    \begin{align*} 
        \EE{\|S_n(\mathbf{t})\|^{2+\delta}} 
        & = \left(\frac{1}{n^{d/2}}\right)^{2+\delta} \EE{\left\| \sum\limits_{\usbf[1] \leq \mathbf{j} \leq \floor{n \mathbf{t}}} X_\mathbf{j} \right\|^{2+\delta} } \\
        & \leq \left(\frac{1}{n^{d/2}}\right)^{2+\delta} C_{d, 2+\delta} \left(\prod\limits_{l = 1}^d \floor{n \mathbf{t}_l}\right)^{\frac{2+\delta}{2}} \\
        & = C_{d, 2+\delta} \left(\prod\limits_{l = 1}^d \frac{\floor{n \mathbf{t}_l}}{n}\right)^{\frac{2+\delta}{2}} \\
        & \leq C_{d, 2+\delta} [\mathbf{t}]^{\frac{2+\delta}{2}} \\
        & \leq C_{d, 2+\delta}
    \end{align*}
    for all $n$ and $\mathbf{t}$. Therefore
    \[ \sup\limits_n \EE{\|S_n(\mathbf{t})\|^{2+\delta}} \leq C_{d, 2+\delta} < \infty. \]
    In \cite{[4]billingsley1968convergence} p. 32 it is remarked that this implies the uniform integrability Assumption c).

    To show Condition d), let $B_j = (\mathbf{s}^j, \mathbf{t}^j]$, $j = 1, ..., q$, be a collection of strongly separated blocks. Without loss of generality (after reordering the blocks) we may assume that there is an index $i \in \{1, ..., d\}$ such the blocks are ordered in the sense that
    \[ 0 \leq \mathbf{s}^1_i \leq \mathbf{t}^1_i < ... < \mathbf{s}^q_i \leq \mathbf{t}^q_i \leq 1, \]
    implying $\min_{j=1, ..., q-1}(\mathbf{s}_{i}^{j+1} - \mathbf{t}_i^{j}) > 0$ and thus
    \begin{equation} \label{lemma3: min difference diverging}
        r_n \coloneqq \min\limits_{j = 1, ..., q-1} \left( \floor{n \mathbf{s}_i^{j+1}} - \floor{n \mathbf{t}_i^{j}} \right) \to \infty \ \text{for } n \to \infty. 
    \end{equation}
    %See also Remark \ref{remark:strongly separated blocks}. 
    Now let $H_j$, $j = 1, ..., q$ be a collection of Borel-sets in $\RR^k$. Then 
    \[ \left| \PP\left( S_n(B_j) \in H_j \ \forall j \right) - \prod\limits_{j=1}^q \PP\left(S_n(B_j) \in H_j\right) \right| \]
    is equal to the telescoping sum $\left|\sum\limits_{l=1}^{q} (a_{l} - a_{l-1})\right|$ with
    \[ a_l \coloneqq \PP\left( S_n(B_j) \in H_j \ \forall j = 1, ..., l \right) \prod\limits_{j=l+1}^{q} \PP\left( S_n(B_j) \in H_j \right) \]
    for $l = 0, ..., q$. Setting the events
    \[ A_l \coloneqq \left\{ \bigcap\limits_{j = 1}^{l-1} S_n(B_j) \in H_j \right\}, \ B_l \coloneqq \left\{ S_n(B_{l}) \in  H_{l} \right\} \]
    for $l > 0$ and $A_0 \coloneqq B_0 \coloneqq \Omega$, we may write
    \[ a_l - a_{l-1} = \PP\left( A_l \cap B_l \right) \prod\limits_{j=l+1}^{q} \PP\left( S_n(B_j) \in H_j \right) - \PP(A_l) \PP(B_l) \prod\limits_{j=l+1}^{q} \PP\left( S_n(B_j) \in H_j \right). \]
    By definition of $r_n$, for each $l$ there are, in the spirit of Definition \ref{defn:mixing coefficients} of the $\rho_\RR$-mixing coefficient,
    \[ M_l, N_l \subset \ZZ^d: \exists A, B \subset \ZZ, \mathrm{dist}(A, B) \geq r_n: \forall \mathbf{j} \in M, \mathbf{k} \in N: \mathbf{j}_i \in A, \mathbf{k}_i \in B \]
    such that $A_l \in \mathcal{A}_M \coloneqq \sigma(X_\mathbf{k}: \mathbf{k} \in M), B_l \in \mathcal{A}_N$.
    Therefore we can bound the telescoping sum as follows:
    \begin{align*}
        \left|\sum\limits_{l=1}^{q} (a_{l} - a_{l-1})\right|
        & \leq \sum\limits_{l=1}^{q} \prod\limits_{j=l+1}^{q} \PP\left( S_n(B_j) \in H_j \right) \left| \PP\left( A_l \cap B_l \right)  - \PP(A_l) \PP(B_l) \right| \\
        & \leq \sum\limits_{l=1}^{q} \left| \PP\left( A_l \cap B_l \right)  - \PP(A_l) \PP(B_l) \right| \\
        & \leq \sum\limits_{l=1}^q \alpha(\mathcal{A}_{M_l}, \mathcal{A}_{N_l}).
    \end{align*}
    According to \cite{bradley1986basic} Inequality (1.8), this is bounded from above by
    \begin{align*}
        \sum\limits_{l=1}^q  \frac{1}{4} \rho_\RR(\mathcal{A}_{M_l}, \mathcal{A}_{N_l}) 
        & \leq \frac{q}{4} \rho_\RR(r_n).
    \end{align*}
    By the $\rho$-mixing Assumption \ref{assumption:rho_mixing} in conjunction with \eqref{lemma3: min difference diverging}, this tends to $0$ as $n$ goes to infinity. This means that we have shown 
    \[ \lim\limits_{n \to \infty} \PP\left( S_n(B_j) \in H_j \ \forall j \right) - \prod\limits_{j=1}^q \PP\left(S_n(B_j) \in H_j\right) = 0 \]
    for strongly separated blocks $B_j$ and Borel-sets $H_j$, which is Condition d) of Lemma \ref{lemma:2}.

    To show e), note that we may, analogously to the proof of Condition c), apply Lemma \ref{lemma:1} to get
    \[ \EE{M(U)^{2+\delta}} \leq \Tilde{C}_{d, 2+\delta} |U|^{\frac{2+\delta}{2}} \]
    from Inequality \eqref{lemma1:(9)} for any discrete block $U \subset \NN^d$. Rewriting this, we get
    \[ \EE{\left(\frac{M(U)^2}{|U|}\right)^{\frac{2+\delta}{2}}} \leq \Tilde{C}_{d, r} < \infty \]
    where the upper bound does not depend on $U$. Because of $\frac{2+\delta}{2} > 1$, this shows the uniform integrability of any family
    \[  \left\{\left(\frac{M(U_n)^2}{|U_n|}\right)^{\frac{2+\delta}{2}}: n \in \NN \right\} \]
    of discrete blocks $U_n \subset \NN^d$. In the proof of \cite{[11]bulinksi2007limittheorems} Theorem 1.3. in Chapter 5 it is concluded that this ensures the condition e) regarding the modulus of continuity of the partial sum fields for univariate random fields. The proof does not change when considering multivariate ones. Note that both \cite{[11]bulinksi2007limittheorems} Theorem 1.3. and Lemma \ref{lemma:1} consider random fields indexed by $\NN^d$ instead of the usual $\ZZ^d$.

    We have now shown all conditions of Lemma \ref{lemma:2}. Applying that lemma concludes the proof.
\end{proof}


\begin{remark}
    The constants $(2+\delta)/\delta$ and $2+\delta$ in the above proof stem from the covariance inequality
    \[ |Cov(Y, Z) \leq 2 \alpha(Y, Z)^{1/p} \|Y\|_q \|Z\|_r \]
    with positive constants $p, q, r$ such that 
    \[ p^{-1} + q^{-1} + r^{-1} = 1 \]
    where
    \[ \alpha(Y, Z) \coloneqq 2 \sup\limits_{y, z \in \RR} | \PP(Y > y, Z > z) - \PP(Y > y)\PP(Z > z), \]
    see \cite{[46]rio2013inequalities} (1.12b).
\end{remark}
