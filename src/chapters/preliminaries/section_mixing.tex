\section{Mixing}

Intuitively, mixing coefficients measure how much mutual information different observations of a random field carry, depending on how far apart they are in space.
The question how to measure distance does not have many different answers in the case of stochastic processes, i.e., the index set being $\ZZ$. In this case, one only needs to define wheter or not index sets are allowed to be "interlaced", see \cite{bradley2001interlaced} for a demonstration of the difference between the two.
However, when moving to random fields, one has further options which are expressed in the two mixing coefficients $\rho$ and $\rho^*$. In addition to these two, we are also interested in the $\alpha$-mixing coefficients which has been introduced in \cite{rosenblatt1956central}. For an overview on some other mixing coefficients that are used in the literature, see \cite{bradley1986basic} and Bradley's other work.

\begin{defn}[Mixing coefficients] \label{defn:mixing coefficients}
    The mixing coefficients of two $\sigma$-fields $\mathcal{A}$ and $\mathcal{B}$ are defined via
    \begin{equation}
        \alpha(\mathcal{A}, \mathcal{B}) \coloneqq \sup\limits_{\substack{A \in \mathcal{A} \\ B \in \mathcal{B}}} | \PP(A \cap B) - \PP(A) \PP(B)|
    \end{equation}
    and
    \begin{equation}
        \rho_H(\mathcal{A}, \mathcal{B}) \coloneqq \sup\left\{ \frac{\EE{\langle Y, Z \rangle} - \langle \EE{Y}, \EE{Z} \rangle}{\|Y\|_2 \|Z\|_2} \mid \substack{Y \in L^2(\mathcal{A}, H), Z \in L^2(\mathcal{B}, H), \\ \Var(Y), \Var(Z) > 0 } \right\}.
    \end{equation}
    Now let $X$ be a random field and define the $\sigma$-algebra generated by the observations $X$ for an index set $M \subset \ZZ^d$ as
    \[ \mathcal{A}_M \coloneqq \sigma(X_{\mathbf{j}}: \mathbf{j} \in M). \]
    Then the mixing coefficients of $X$ are defined as the functions given by
    \begin{equation}
        \gls*{alpha}_{k, m}(r) = \sup\{ \alpha(\mathcal{A}_M, \mathcal{A}_N) \mid M, N \subset \ZZ^d, \mathrm{dist}(M, N) \geq r, |M| \leq k, |N| \leq m \},
    \end{equation}
    \begin{equation} \label{defn:rho_mixing coefficient} \begin{split}
        \rho_H(r) \coloneqq \sup\{ \rho_H(\mathcal{A}_M, \mathcal{A}_N) \mid & M, N \subset \ZZ^d, \exists i \in \{1, ..., d\} \exists A, B \subset \ZZ, \mathrm{dist}(A, B) \geq r: \\ 
        & \forall \mathbf{j} \in M, \mathbf{k} \in N: \mathbf{j}_i \in A, \mathbf{k}_i \in B \},
    \end{split} \end{equation}
    \begin{equation} \label{defn:rho_star_mixing coefficient}
        \rho^*_H(r) \coloneqq \sup\{ \rho_H(\mathcal{A}_M, \mathcal{A}_N) \mid M, N \subset \ZZ^d, \mathrm{dist}(M, N) \geq r \} 
    \end{equation}
    where $\mathrm{dist}$ denotes the distance w.r.t. the supremum norm.

    A random field is said to be \textit{$c$-mixing} for some given coefficient function $c$ (here $\rho$, $\rho^*$ or $\alpha_{k,m}$) if $\lim\limits_{r \to \infty} c(r) = 0$.
\end{defn}

\begin{remark} \label{remark:mixing coefficients coincide}
    As was shown in \cite{[6]BRADLEY1985335} Theorem 4.2, the $\rho_H$-coefficient does not depend on what space $H$ one chooses. For this reason, we will usually write $\gls*{rho}$ instead of $\rho_H$ from now on. One may always think about the case $H = \RR$, in which case the definition of the $\rho$-mixing coefficient measures the maximum absolute correlation.
\end{remark}

According to \cite{[0]BUCCHIA2017344}, mixing conditions are very common in the literature even though they "are not easy to verify in practice". For the exact assumptions we are going to impose on the mixing coefficients, see Section \ref{section:assumptions}.

\begin{remark} \label{rem:mixing coefficients smaller for transform}
    Let $X$ be a random field and $T(X)$ a function of it. Then the mixing coefficients of $T(X)$ are no bigger than the ones of $X$ as the $\sigma$-algebras generated by $T(X)$ are contained in the ones generated by $X$ and therefore the suprema in the definitions of the mixing coefficients are taken over smaller sets.
\end{remark}
