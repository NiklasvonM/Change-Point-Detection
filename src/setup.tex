\documentclass[12pt, a4paper, oneside]{book} %book, report, thesis % oneside, twoside
%\documentclass[12pt, a4paper,twoside, openany]{memoir}


% https://tex.stackexchange.com/questions/8458/making-the-bibliography-appear-in-the-table-of-contents
\usepackage[nottoc,numbib]{tocbibind}

\usepackage[utf8]{inputenc}
\usepackage[T1]{fontenc}

%to set the page margin:
\usepackage[a4paper]{geometry} %[scale=0.75,top=1cm] [a4paper]


% Packages
%\usepackage[latin1]{inputenc}
\usepackage{graphicx} % for \includegraphics
%\usepackage{dsfont}
\usepackage{amsmath}
\usepackage{mathtools}
\usepackage{bbm} % for \mathbbm{1}
\usepackage{mathrsfs} % \mathscr{S} old German letters
%\usepackage[ansinew]{inputenc}
%\usepackage[round, sort, comma, numbers]{natbib}
%\usepackage[dvipsnames]{xcolor}
\usepackage{amssymb}
% linguistic package
%\usepackage[ngerman]{babel}
\usepackage[english]{babel}
\usepackage{csquotes} %The "csquotes" package is also used to provide proper formatting of quotes in the bibliography. With these commands in place, you should be able to use the \bibliographystyle{alphadin} in English in your LaTeX document.
%\usepackage{float}
%\usepackage{bbm}
%\usepackage{gensymb}
%\usepackage{makeidx}
%\usepackage{showidx} %makes indices appear at the top right, don't use
\usepackage{amsthm} % for environment proof (\begin{proof})
\usepackage{xfrac} % for diagonal lines in factor groups (\sfrac)
\usepackage{faktor} % for diagonal lines in factor groups (\faktor)
\usepackage[noadjust]{cite} % BibTeX https://www.youtube.com/watch?v=KS9GvK7cvmo
\usepackage{svg} % https://tex.stackexchange.com/questions/122871/include-svg-images-with-the-svg-package
\usepackage{import} %\import{}
\usepackage[noadjust]{cite} % BibTeX https://www.youtube.com/watch?v=KS9GvK7cvmo

\usepackage{tikz} % for diagrams
\usepackage{tikz-cd}
 % to not start arrows in the middle 
 % https://tex.stackexchange.com/questions/483035/having-arrows-enter-and-leave-from-different-parts-of-a-node-in-tikzcd
\tikzstyle{start}=[to path={(\tikztostart.#1) -- (\tikztotarget)}]


\usepackage{hyperref}% http://ctan.org/pkg/hyperref
% (\cref) https://tex.stackexchange.com/questions/58367/reference-lemma-from-the-same-document
%\usepackage{cleveref} % for reference within document
%\usepackage{showlabels} % to show labels. TODO for final version: comment out/remove! %[inline]

\usepackage{lmodern}% http://ctan.org/pkg/lm remove warning "Size substitutions with differences..."

\usepackage[toc,page]{appendix}



\newcommand{\comment}[1]{}

\DeclareMathOperator{\Ima}{Im} % Im()


%\usetikzlibrary{babel} % for compability between ngerman babel and tikz-cd

%%% For incscape
%\usepackage{import}
%\usepackage{xifthen}
%\usepackage{pdfpages}
%\usepackage{transparent}

\newcommand{\incfig}[1]{%
	\def\svgwidth{0.7\columnwidth}
	\import{./Graphiken/}{#1.pdf_tex}
}
%%%


%%% https://be-jo.net/2013/08/latex-welchen-bibliographystyle-wahlen/
%\bibliographystyle{alphadin} % Gut, aber deutsch mit S.233
%\bibliographystyle{unsrt} % [1], keine Links
%\bibliographystyle{chicago} % führt zu Fehlern
%\bibliographystyle{apalike} % [Adler, 2021], keine Links
%\bibliographystyle{plain} % zu wenig Informationen, [1]
\bibliographystyle{alpha} % 'Gut, aber keine Links
%\bibliographystyle{natbib} % Fehler


\newcommand{\TODO}{\textcolor{blue}{TODO\ }}
\newcommand{\open}[1]{
	\overset{\circ}{#1}
}

% 3 dimensional row vector
\newcommand{\vct}[3]{\begin{pmatrix} #1 \\ #2 \\ #3 \end{pmatrix}}



% tikz { http://ctan.math.washington.edu/tex-archive/graphics/pgf/contrib/tikz-cd/tikz-cd-doc.pdf

%\usetikzlibrary{cd}

% tikz }

\usepackage{adjustbox} % \adjustbox{scale = ...}{}







% Definitionen
\def\n{\newline}

\makeindex

\theoremstyle{plain}
\newtheorem{thm}{Theorem}[section] % reset theorem numbering for each chapter
\newtheorem{lemma}[thm]{Lemma}
\newtheorem{corollary}[thm]{Corollary}
\newtheorem{proposition}[thm]{Proposition}

\theoremstyle{definition}
\newtheorem{defn}[thm]{Definition} % definition numbers are dependent on theorem numbers
%\newtheorem{satzdef}[thm]{Satz und Definition}

\theoremstyle{remark}
\newtheorem{ex}[thm]{Example} % same for example numbers
\newtheorem{remark}[thm]{Remark}
\newtheorem{conv}[thm]{Convention}

% https://tex.stackexchange.com/questions/21598/how-to-color-math-symbols
\makeatletter
\def\mathcolor#1#{\@mathcolor{#1}}
\def\@mathcolor#1#2#3{%
  \protect\leavevmode
  \begingroup
    \color#1{#2}#3%
  \endgroup
}
\makeatother

% Section name on top of each page
%\pagestyle{headings}


%%%%%%%%%%%%%%%%%%%%%%%%%%%%%%%%%%%%%%%%%%
% Without this, the page number for \documentclass{thesis} is not showing at the first page of a chapter.
% https://tex.stackexchange.com/questions/527589/page-numbering-not-on-first-page-of-each-chapter
%\usepackage{fancyhdr} 
%\pagestyle{headings} % options: fancy, headings, myheadings
%\fancypagestyle{plain}{
%    \fancyhf{} % clear all header and footer fields 
%    \fancyhead[RO,LE]{\thepage} % Right Odd, Left Even => Outside
    % Some examples
    %\fancyhead[R]{\thepage} % Right side has the page number
    %\fancyhead[L]{\leftmark} % Left side has the \chapter{title} 
%    \renewcommand{\headrulewidth}{0pt} % remove line between header and main text 
%}
%%%%%%%%%%%%%%%%%%%%%%%%%%%%%%%%%%%%%%%%%




%%%%%%%%%%%%%%%%%%%%%%%%%%%%%%%
% Change chapter command to show page number also on the first page of a chapter
% https://tex.stackexchange.com/questions/103567/how-to-remove-page-numbers-from-first-page-of-chapters
\makeatletter
\renewcommand\chapter{\if@openright\cleardoublepage\else\clearpage\fi
                    \thispagestyle{plain}%
                    \global\@topnum\z@
                    \@afterindentfalse
                    \secdef\@chapter\@schapter}
\makeatother
%%%%%%%%%%%%%%%%%%%%%%%%%%%%%%%



% (a), (b), (c)
\newcounter{aufzi}
\newenvironment{aufzi}{\begin{list}{ {\upshape\alph{aufzi})}}{
        \usecounter{aufzi}
        \topsep1ex
%        \partopsep
        \parsep0cm
        \itemsep1ex
        \leftmargin0.8cm
%        \rightmargin
%        \listparindent
        \labelwidth0.5cm
        \labelsep0.3cm
        %\itemindent-0.3cm
}}
{\end{list}}

% (i), (ii), (iii)
\newcounter{aufzii}
\newenvironment{aufzii}{\begin{list}{\hfill {\upshape 
(\roman{aufzii})}}{
        \usecounter{aufzii}
        \topsep1ex
%        \partopsep
        \parsep0cm
        \itemsep1ex
        \leftmargin0.8cm
%        \rightmargin
%        \listparindent
        \labelwidth0.5cm
        \labelsep0.3cm
         %\itemindent0.3cm
}}
{\end{list}}


%\allowdisplaybreaks


% https://tex.stackexchange.com/questions/118173/how-to-write-ceil-and-floor-in-latex
\DeclarePairedDelimiter\ceil{\lceil}{\rceil}
\DeclarePairedDelimiter\floor{\lfloor}{\rfloor}

\newcommand{\RR}{\mathbb{R}}
\newcommand{\NN}{\mathbb{N}}
\newcommand{\ZZ}{\mathbb{Z}}
\newcommand{\QQ}{\mathbb{Q}}
\newcommand{\CC}{\mathbb{C}}

%\newcommand{\EE}{\mathbb{E}}
\newcommand{\EE}[1]{\mathbb{E}\left[#1\right]}
\newcommand{\PP}{\mathbb{P}}
\newcommand{\Var}{\mathrm{Var}}
\newcommand{\Cov}{\mathrm{Cov}}
\newcommand{\spn}{\mathrm{span}} 

\newcommand{\limToInf[1]}{\lim\limits_{#1 \to \infty}}

\newcommand{\usbf[1]}{\ensuremath{\mathbf{\underline{#1}}}}



% Hypothesis testing
\newtheorem{innerhyp}{Hypothesis}
\newenvironment{hyp}[1]{%
  \renewcommand\theinnerhyp{#1}\innerhyp
}{\endinnerhyp}

%%%%%%%%%%%%%% Assumptions
\newtheorem{assumption}{Assumption}

% Alternative Assumption!
\newtheorem{assumptionalt}{Assumption}[assumption]

% New environment
\newenvironment{assumptionp}[1]{
  \renewcommand\theassumptionalt{#1}
  \assumptionalt
}{\endassumptionalt}



%%%%%%%%%%%%%%%%%%%%%%%%%%%%%%%%%%%%%%%%%%%%%%%%%%%%%%%%%%%%%%%%%
%%%%%%%%%%%%%%%%%%%%%% Glossaries %%%%%%%%%%%%%%%%%%%%%%%%%%%%%%%
%\usepackage{glossaries}
\usepackage[symbols,nogroupskip]{glossaries-extra} %,nonumberlist


\makeglossaries

\glsxtrnewsymbol[description={Long-run variance}]{Gamma}{\ensuremath{\Gamma}}
\glsxtrnewsymbol[description={auto-covariance function}]{gamma}{\ensuremath{\gamma}}
\glsxtrnewsymbol[description={dimension of the index set}]{d}{\ensuremath{d}}
\glsxtrnewsymbol[description={random field}]{X}{\ensuremath{X}}
\glsxtrnewsymbol[description={separable Hilbert space}]{H}{\ensuremath{H}}
\glsxtrnewsymbol[description={expected value}]{EE}{\ensuremath{\mathbb{E}}}
\glsxtrnewsymbol[description={set of integers}]{ZZ}{\ensuremath{\ZZ}}
\glsxtrnewsymbol[description={set of natural numbers (without $0$)}]{NN}{\ensuremath{\NN}}
\glsxtrnewsymbol[description={$\alpha$-mixing coefficient}]{alpha}{\ensuremath{\alpha}}
\glsxtrnewsymbol[description={$\rho$-mixing coefficient}]{rho}{\ensuremath{\rho}}
\glsxtrnewsymbol[description={probability measure}]{PP}{\ensuremath{\PP}}
\glsxtrnewsymbol[description={sample mean $\frac{1}{n} \sum_{i=1}^n X_i$}]{Xbar}{\ensuremath{\overline{X}_n}}
\glsxtrnewsymbol[description={Gaussian brackets $\floor*{x}=\max\{j \in \ZZ \mid j \leq x\}$}]{floor}{\ensuremath{\floor*{\cdot}}}
\glsxtrnewsymbol[description={convergence in distribution, weak convergence}]{weakconvergence}{\ensuremath{\Rightarrow}}
%\glsxtrnewsymbol[description={Skhorokod space (of càdlàg functions $[0, 1] \to H$)}]{D01}{\ensuremath{\mathcal{D}_H[0,1]}}
\glsxtrnewsymbol[description={expected value}]{mu}{\ensuremath{\mu}}
\glsxtrnewsymbol[description={Lebesgue measure}]{lambda}{\ensuremath{\lambda}}
\glsxtrnewsymbol[description={Kronecker delta}]{deltaij}{\ensuremath{\delta_{i, j}}}
\glsxtrnewsymbol[description={number of elements in a set}]{hashtag}{\ensuremath{\#}}
\glsxtrnewsymbol[description={set of bounded linear operators $V \to W$}]{LVW}{\ensuremath{\mathcal{L}(V, W)}}
\glsxtrnewsymbol[description={vector containing x in each of its components}]{usbfx}{\ensuremath{\usbf[x]}}
\glsxtrnewsymbol[description={\ensuremath{\prod_i \mathbf{x}_i}}]{multiplicationx}{\ensuremath{[x]}}
\glsxtrnewsymbol[description={operator norm of $\varphi$}]{operatornormvarphi}{\ensuremath{\| \varphi \|_{\mathrm{op}}}}
\glsxtrnewsymbol[description={belonging standardly to}]{lhd}{\ensuremath{\lhd}}
\glsxtrnewsymbol[description={(algebraic) tensor product}]{otimes}{\ensuremath{\otimes}}
\glsxtrnewsymbol[description={(complete) tensor product}]{hatotimes}{\ensuremath{\hat{\otimes}}}
\glsxtrnewsymbol[description={modulus of continuity}]{w}{\ensuremath{w}}
\glsxtrnewsymbol[description={dash-product of sets}]{circleddash}{\ensuremath{\circleddash}}


% https://ctan.joethei.xyz/macros/latex/contrib/glossaries/glossaries-user.pdf page 66
%\renewcommand*{\glsclearpage}{}
%\renewcommand*{\glsclearpage}{\clearpage}

% https://tex.stackexchange.com/questions/10924/underfull-hbox-in-bibliography
\usepackage{etoolbox}
\apptocmd{\thebibliography}{\raggedright}{}{}
